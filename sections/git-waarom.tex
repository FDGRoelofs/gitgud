\section{Waarom \& hoe git?}

\subsection{Version control}
\begin{frame}{Wat is version control?}
	Op een niet-onhandige manier:
		\begin{itemize}
			\item Code delen met meerdere mensen
			\item Geschiedenis bijhouden / terugdraaien
			\item Veranderingen (evt. over langere tijd) zichtbaar maken
			\item Overzicht van wie wat veranderde
		\end{itemize}
\end{frame}

\subsection{Waarom git?}
\begin{frame}{Waarom git?}
	\begin{itemize}
		\item Distributed:
			\begin{itemize}
				\item Je hebt zelf (standaard) de gehele geschiedenis
				\item Je kan zonder internet werken
			\end{itemize}
		\item Integrity: bestanden kunnen niet veranderen zonder dat git het merkt
		\item Moeilijk om data te vernietigen (`kwijt' kan maar dan moet je beter opletten)
		\item Non-lineair: kan veel dingen naast elkaar starten (en weer weggooien)
		\item Veel diensten op gebouwd: Github, Travis, \ldots
	\end{itemize}
\end{frame}

\begin{frame}{Waarom command line?}
	\begin{itemize}
		\item Hetzelfde over alle platforms
		\item GUIs gebruiken zelfde namen voor dingen
	\end{itemize}
	Goede GUIs:
	\vspace{15mm}
	\begin{tabular}{ll}
		Github desktop (Win, Mac)&\url{http://desktop.github.com}\\
		Sourcetree (Win, Linux)&\url{http://sourcetreeapp.com}
	\end{tabular}
\end{frame}
