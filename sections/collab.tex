\section{Samenwerken / remotes}

\subsection{Remotes}
\begin{frame}{Remotes}
	Remote: andere (volledige) git repo met gedeeld punt in geschiedenis
	\begin{itemize}
		\item \texttt{git remote -v}: toon remotes met url
		\item \texttt{git remote add <naam> <url>}: \texttt{naam} wordt alias voor \texttt{url}
	\end{itemize}
	Goede plekken om je repository te hosten:
	\begin{itemize}
		\item Github: veel gebruikt, Travis, gratis student account (private repo's beperkt)
		\item Bitbucket: alternatief voor Github, onbeperkt private repo's
		\item UU GitLab: intern voor de UU, mailinglijst, onbeperkt private repo's
	\end{itemize}
\end{frame}

\subsection{SSH keys}
\begin{frame}{SSH keys}
	\begin{enumerate}
		\item Open terminal/Terminal/git-bash
		\item \texttt{ssh-keygen}
		\item Edit je \texttt{.bashrc} of \texttt{.profile}\\
			Voeg toe: \texttt{eval `ssh-agent`}
	\end{enumerate}
\end{frame}

\subsection{Bestaande repo overnemen}
\begin{frame}{git clone}
	Kopieer repository en maak remote `origin'
	\begin{itemize}
		\item \texttt{git clone <url>}
		\item \texttt{git clone <url> <map>}
	\end{itemize}
	Als niet je eigen repo en wel toevoegingen maken: forken
\end{frame}

\subsection{Wijzigingen binnenhalen}
\begin{frame}{git fetch}
	Kopieer commits van een remote naar je repo
\end{frame}
