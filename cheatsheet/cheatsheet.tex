\documentclass[9pt,a4paper]{extarticle}

\usepackage[dutch]{babel}
\usepackage{fourier}
\usepackage[cm,empty]{fullpage}
\usepackage{xcolor}
\usepackage{multicol}
\usepackage{minted}
\usepackage{hyperref}
\usemintedstyle{vs}

\title{Cheatsheet Git}
\author{Sticky: CommIT}
\date{}

\setlength{\parindent}{0pt}
\setlength{\parskip}{0pt}

\newcommand{\shell}[1]{\mintinline{bash}{#1}}
\newcommand{\lett}[1]{\texttt{#1}}

\begin{document}
Let op: plekken in commando's waar je zelf iets in moet vullen worden aangegeven met \shell{$waarde}. Als je dit invoert moet je
de \shell{$} niet overnemen. Opties in \shell{[ ]} blokhaken zijn optioneel.

\section*{Instellingen}
Stel in met \shell{git config --global $optie $waarde}. Laat \shell{--global} weg om alleen in het huidige repository de
instelling aan te passen. Lees alle wijzigingen (die de commando's aanbrengen)  met \shell{less ~/.gitconfig} .

\newlength{\OptieBreed}
\settowidth{\OptieBreed}{\shell{commit.verbose} }
\begin{tabular}{p{\OptieBreed}p{\linewidth-\OptieBreed}}
	\textbf{Optie}			& \textbf{Wat doet het}\\ \hline
	\shell{user.name}		& Je naam, zichtbaar in alle commits.\\
	\shell{user.email}		& Email, zichtbaar in alle commits.\\
	\shell{core.editor} 	& Editor voor commit messages. Zet dit op \texttt{nano} voor het minste gedoe.\\
	\shell{color.ui}		& Zet op \shell{auto} voor gekleurde tekst.\\
	\shell{commit.verbose}  & Zet op \shell{true} voor een overzicht van alle veranderingen bij een commit message schrijven.
\end{tabular}

\section*{Basisacties}
\settowidth{\OptieBreed}{\shell{git reset HEAD $file1 [$file2..]}}
\begin{tabular}{p{\OptieBreed}p{\linewidth-\OptieBreed}}
	\textbf{Commando}					& \textbf{Wat doet het}\\ \hline
	\shell{git status}					& Geef algemene informatie over de toestand van je repository.							\\
	\shell{git init}					& Maak nieuw, leeg repository in huidige map.											\\
	\shell{git clone $url}				& Clone het repository dat kan worden gevonden op \shell{$url}. 						\\
	\shell{git add $file1 [$file2..]} 	& Neem bestanden op in de volgende commit. (Zowel wijzigingen als nieuwe bestanden.)	\\
	\shell{git reset HEAD $file1 [$file2..]} & Maak \shell{git add} ongedaan voor gegeven bestanden.							\\
	\shell{git reset}					& Maak alle \shell{git add}'s ongedaan.													\\
	\shell{git commit}					& Maak een commit aan, opent je ingestelde editor voor een bericht.						\\
	\shell{git commit -m $tekst}		& Maak een commit aan met een vooringesteld bericht. (Let op aanhalingstekens rond je
											bericht!)																			\\
	\shell{git commit --amend}			& Pas een zojuist gemaakte commit aan. \textbf{Doe dit alleen bij ongedeelde commits!}
\end{tabular}

\section*{Bestanden negeren}
Je kan git aangeven bestanden te negeren door een bestand met filters aan te maken:
\begin{itemize}
	\item in een map in je repository, met als naam \shell{.gitignore}, voor iedereen die het repository gebruikt en voor alle
		submappen en bestanden in de map waar het bestand staat.
	\item in \shell{$repo/.git/info} (verborgen map), met als naam \shell{exclude}, voor jou alleen, automatisch voor het gehele repository.
	\item in \shell{~/.config/git}, met als naam \shell{exclude}, voor jou alleen, voor alle repositories op die computer.
\end{itemize}
Zie \shell{git help ignore} voor alle mogelijkheden met filters.
\begin{minted}{bash}
bin/*       # Negeer alle bestanden in de map bin
*.exe       # Negeer alle bestanden die eindigen op .exe, overal
!dinges.exe # ...Maar negeer dinges.exe niet
\end{minted}

\section*{Geschiedenis}
\settowidth{\OptieBreed}{\shell{git diff $id1..$id2 }}
\begin{tabular}{p{\OptieBreed}p{\linewidth-\OptieBreed}}
	\shell{git diff}			& Toon alle verschillen tussen je (niet-gestagede) bestanden en de laatste commit.\\
	\shell{git diff --staged} 	& Toon wat er klaarstaat om gecommit te worden.\\
	\shell{git diff $id1..$id2} & Toon de precieze verschillen tussen commit \shell{$id1} en \shell{$id2}.\\
	\shell{git diff $id}		& Toon de precieze verschillen tussen commit \shell{$id} en de nieuwste (\shell{HEAD}).\\
	\shell{git show}			& Toon informatie over de laatste commit.\\
	\shell{git log}				& Geef een lijst van commits, nieuwste eerst.\\
	\shell{git log --reverse} 	& Geef een lijst van commits, oudste eerst.\\
	\shell{git show $id}		& Toon informatie over commit \shell{$id}.
\end{tabular}

\section*{Terugdraaien}
\textbf{Denk na wat je doet! De commando's gemerkt met \danger{} gooien data weg!}

\settowidth{\OptieBreed}{\shell{git checkout HEAD $bestand }}
\begin{tabular}{p{\OptieBreed}p{1ex}p{\linewidth-\OptieBreed-1ex}}
	\shell{git checkout -- $bestand} 		& \danger{} & Zet \shell{$bestand} terug naar de versie zoals in je
	laatste commit.\\
	\shell{git checkout $id $bestand}		& & Zet \shell{$bestand} terug naar de versie zoals in commit \shell{$id}.\\
	\shell{git checkout HEAD $bestand}		& & Zet \shell{$bestand} terug naar de verzie zoals in je laaste commit.\\
	\shell{git revert $id}					& & Maak een nieuwe commit die de wijzigingen in \shell{$id} ongedaan maakt.\\
	\shell{git reset --hard}				& \danger{} & \shell{git checkout --} op al je bestanden. (!)\\
\end{tabular}

Tip: Bij de meeste dingen waar je meerdere bestanden op mag geven kan je ook een mapnaam opgeven, om
alle bestanden in deze map in \'e\'en keer aan te merken. \texttt{.} slaat op de huidige map.

\section*{Untracked bestanden opruimen}
Met \shell{git clean} kan je bestanden die git niet bijhoudt verwijderen. \textbf{Denk dus na wat je doet!} \shell{git
clean -n} toont je wat een `echte' clean zou doen, met \shell{-x} en \shell{-d} kan je genegeerde bestanden en
genegeerde mappen ook verwijderen. \shell{git clean -f [-x] [-d]} voert de verwijdering zonder waarschuwing uit.
(\danger{})

\section*{Branching}
\settowidth{\OptieBreed}{\shell{git checkout --no-merged }}
\begin{tabular}{p{\OptieBreed}p{1ex}p{\linewidth-\OptieBreed-1ex}}
	\shell{git branch $naam}		&& Maak een nieuwe branch, \shell{$naam}, aan.\\
	\shell{git checkout $naam}		&& Stel branch \shell{$naam} in als actief (bestanden in map zijn van die branch).\\
	\shell{git checkout -b $naam}	&& Maak een branch en stel in als actieve branch.\\
	\shell{git merge $naam}			& \danger{} & Neem alle wijzigingen in branch \shell{$naam} over in de huidige branch.
										\textbf{Eerst committen!}\\
	\shell{git branch}				&& Toon alle lokale branches.\\
	\shell{git branch --no-merged}  && Toon alle branches die commits bevatten die niet in de huidge branch zitten.\\
	\shell{git branch -d $naam}		&& Verwijder een gemergede branch.
\end{tabular}

\subsection*{Merge conflict}
Wanneer wijzigingen in een merge elkaar tegenspreken (bestand in beide branches gewijzigd) heb je een \emph{merge conflict}:
\begin{enumerate}
	\item Doe \shell{git status} en kijk welke bestanden conflicteren.
	\item Open de bestanden en zoek naar \shell{<<<<<}, \shell{=====} of \shell{>>>>>}. (Dit zijn de gebieden die
		conflicteren.) Los hier het probleem op en verwijder de \shell{<<<}, \shell{===} en \shell{>>>}-regels.
	\item Wanneer alle conflicten zijn opgelost kan je de bestanden \shell{git add}en, en committen.
\end{enumerate}
Dit is niet altijd de handigste manier, vaak kan je beter een visueel programma gebruiken om de conflicten op te lossen.
Programma's die hiervoor werken zijn o.a. Meld (\url{meldmerge.org}), KDiff3 (\url{kdiff3.sourceforge.net}). Voor alle
direct ondersteunde programma's kan je \shell{git mergetool --help-tool} doen.

\section*{Samenwerken}
Een \emph{remote} is een compleet zelfstandig repository met tenminste \'e\'en gedeeld punt in de geschiedenis met jouw
repository.

\settowidth{\OptieBreed}{\shell{git push --set-upstream $remote $branch}}
\begin{tabular}{p{\OptieBreed}p{1ex}p{\linewidth-\OptieBreed-1ex}}
	\shell{git remote -v}				& & Toon alle ingestelde remotes.		\\
	\shell{git remote add $naam $url}	& & Voeg nieuwe remote toe.				\\
	\shell{git clone $url [$map]}		& & Maak een kopie van het repository op \shell{$url} (in de map \shell{$map}
												als opgegeven).\\
	\shell{git fetch}					& & Zet wijzigingen van een \emph{upstream} remote klaar om te mergen.\\
	\shell{git merge $naam/$branch}		& & Merge de ge\shell{fetch}te branch naar je lokale branch.\\
	\shell{git pull}					& & Haal wijzigingen op en merge ze meteen (als mogelijk).\\
	\shell{git push --set-upstream $remote $branch} & & Maak de branch \shell{$branch} aan op \shell{$remote}, en
	publiceer je huidige branch hierop. (Hierna ook de standaardbestemming voor \shell{git push}.)\\
	\shell{git push}					& & Publiceer je wijzigingen in een remote.
\end{tabular}

\section*{Zoeken}
Git bevat een optie om in bestanden naar tekst te zoeken: \shell{git grep $tekst}.

\settowidth{\OptieBreed}{\shell{git grep $tekst $map-of-bestand}}
\begin{tabular}{p{\OptieBreed}p{\linewidth-\OptieBreed-1ex}}
	\shell{git grep $tekst} &
		Zoek in de huidige map en submappen naar \shell{$tekst}. \\
	\shell{git grep -n $tekst} &
		Als vorige, en print naast de bestandsnaam het regelnummer. \\
	\shell{git grep -i $tekst} &
		Zoek hoofdletterongevoelig naar \shell{$tekst}, combineerbaar met -n.\\
	\shell{git grep $tekst $map-of-bestand} &
		Zoek alleen in \shell{$map-of-bestand}.
\end{tabular}

\section*{Aliassen}
Je kan je eigen \shell{git dinges}-commando's maken met \shell{git config --global alias.$aliasnaam "$aliasinhoud"}.
\shell{$aliasinhoud} moet een git-commando zijn (zonder \shell{git} ervoor). Een paar handige voorbeelden (alles begint
met \shell{git config --global}):

\settowidth{\OptieBreed}{\shell{alias.overzicht "log --all --graph --decorate --oneline"}}
\begin{tabular}{p{\OptieBreed}p{\linewidth-\OptieBreed-1ex}}
	\shell{alias.overzicht "log --all --graph --decorate --oneline"} &
		\shell{git overzicht} toont een visueel overzicht van je geschiedenis en hoe je branches erbij staan. \\
	\shell{alias.unstage "reset HEAD"} &
		\shell{git unstage} maakt \shell{git add} ongedaan.\\
	\shell{alias.zoek "grep -nir"} &
		\shell{git zoek $tekst $map} zoekt hoofdletterongevoelig naar \shell{$tekst} in \shell{$map}.
\end{tabular}

\section*{Meer handige trucjes die je zelf uit moet zoeken}
\begin{description}
	\item[Submodules] Zelfstandige git-repositories insluiten in een ander repository (\shell{git help submodule}).
	\item[git stash] Je wijzigingen tijdelijk `opzij' zetten met een handig commando (\shell{git help stash}).
	\item[Travis CI] Wanneer je je code pusht, automatisch tests uitvoeren of andere scripts draaien op een andere
		computer. (\url{travis-ci.com}, gratis voor studenten via \url{education.github.com/pack})
\end{description}

{ \footnotesize
	\section{Bash / terminalcommando's}
	\begin{multicols}{4}
		\begin{itemize}
			\item \shell{ls} - Bestanden in map
			\item \shell{cd} - Open map (\shell{..} = omhoog)
			\item \shell{mkdir} - Maak map
			\item \shell{touch} - Maak leeg bestand
			\item \shell{nano} - Simpele editor
			\item \shell{mv} - Verplaats / hernoem
			\item \shell{cp} - Kopi\"eer
			\item \shell{rm} - Verwijder
			\item \shell{rm -r} - Verwijder map + inhoud
			\item \shell{man iets} - \texttt{man}ual van \texttt{iets}
			\item \shell{cmd > best} - Uitvoer naar \texttt{best}
			\item \shell{echo} - Herhaal (echo) invoer
		\end{itemize}
	\end{multicols}

\end{document}
